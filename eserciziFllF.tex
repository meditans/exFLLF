\documentclass[a4paper,10pt]{article}
\usepackage[utf8]{inputenc}
\usepackage[italian]{babel}

\usepackage{stmaryrd}
\usepackage{amsmath}
\usepackage{amssymb}
\usepackage{amsfonts}
\usepackage{bussproofs}

% Questo per mettere il font Palatino
\renewcommand*\rmdefault{ppl}

\begin{document}
\title{Fondamenti Logici di Linguaggi Funzionali \\ \small{Anno accademico 2013/2014 \\ Esercizi del corso}}
\author{Carlo Nucera}

\maketitle

\section*{Esercizio 1}
\paragraph{Testo}
Capire quali funzioni si possono costruire rinunciando al minimo o ad altri costruttori.

\section*{Esercizio 2}
\paragraph{Testo}
Mostrare che non si può definire la somma senza ricorsione e minimo.

\section*{Esercizio 3}
\paragraph{Testo}
Si scrivano il prodotto e l’elevamento a potenza utilizzando $Z^1$, $S^1$, $P^n_i$, $C$, $R$, e si scrivano la differenza e la divisione utilizzando $Z^1$, $S^1$, $P^n_i$, $C$, $R$, $M$.
\paragraph{Prodotto ed Elevamento a potenza}
\begin{center}
\begin{tabular}{p{5 cm} p{5 cm}}
  $\begin{cases} a \times 0 = 0 \\a \times b' = a + (a \times b) \end{cases}$& $\times^2 = R^2[P^1_1,C^3[S^2,P^3_1,P^3_3]]$
  \\
  $\begin{cases} a \nearrow 0 = 1 \\ a \nearrow b' = a \times (a \nearrow b) \end{cases}$& $\nearrow^2 = R^2[P^1_1,C^3[S^2,P^3_1,P^3_3]]$
\end{tabular}
\end{center}


\section*{Esercizio 4}
\paragraph{Testo}
Mostrare che la funzione di Ackermann non è ricorsiva primitiva.

\paragraph{Preliminari}
Ricordiamo la definizione della funzione di Ackermann:
\begin{center} $ \begin{cases}
    A(0,y) = y+1 \\
    A(x+1,0) = A(x,1) \\
    A(x+1,y+1) = A(x, A(x+1,y))
\end{cases} $ \end{center}
Procederemo a dimostrare che non si tratta di una funzione primitiva ricorsiva, trovando una proprietà che vale per tutte le funzioni primitive ricorsive ma non per la funzione di Ackermann.
A questo scopo, diciamo che una funzione $h:\mathbb{N}^2 \rightarrow \mathbb{N}$ \emph{maggiora} $g:\mathbb{N}^k \rightarrow \mathbb{N}$ se esiste un $b\in \mathbb{N}$ tale che, per ogni $a_1, a_2, \dots, a_n \in \mathbb{N}$,
$$g(a_1,\dots,a_k) < h(b, a_M)$$
dove $a_M = max(a_1, \dots, a_k)$.
Notiamo che una funzione non può maggiorare se stessa. Se così fosse, si avrebbe infatti che:
$$\exists a, \forall x_1, x_2, h(x_1,x_2)<h(a,x_M)$$
dove, al solito, $x_M = max(x_1,x_2)$. Allora, detto $m=max(a,x_M)$, avremmo che:
$$h(a,m) < h(a,max(a,m)) = h(a,m)$$
che è ovviamente assurdo.

Il nostro piano è allora mostrare che le funzioni primitive ricorsive sono tutte maggiorate da $A$ (ovvero, intuitivamente, che la funzione di Ackermann cresce più velocemente di ogni funzione primitiva ricorsiva).
Durante la dimostrazione, useremo la monotonia della funzione di Ackermann in entrambi gli argomenti, ovvero
\begin{center} \begin{tabular}{p{4cm} p{4cm}}
    $A(x,y)<A(x,y+1)$ & $A(x,y)<A(x+1,y)$
\end{tabular} \end{center}
e la seguente disuguaglianza:

\begin{align*}
  A(r,A(s,y)) &< A(r+s,A(s,y)) \\ &< A(r+s,A(r+s+1,y)) \\ &= A(r+s+1,y+1) \\ &\leq A(r+s+2,y)
\end{align*}

Cominciamo allora la dimostrazione:

\paragraph{Zero, Successore, Proiezione} 
\begin{align*}
  Z(n) = 0 &< n+1 = A(0,n) \\
  S(n) = n+1 &< n+2 = A(1,n) \\
  P^k_m(n_1,\dots ,n_k) = x_m \leq x &< x+1 = A(0,x)
\end{align*}

\paragraph{Composizione}
Supponiamo di avere $g^k_1,\dots,g^k_m$ e $h^m$, tutte maggiorate da $A$.
Questo significa che $g^k_i(\overline{x}) < A(r_i, x)$ e $h(\overline{y}) < A(s, y)$, per opportuni valori $r_1,\dots,r_m,s$.
Allora, se $f^k = C[h^m,g^k_1,\dots,g^k_m]$, abbiamo che, detto $g_M(\overline{x}) = max_{i=1\dots m} g_i(\overline{x})$
$$f(\overline{x}) < A(s,g_M(\overline{x})) < A(s,A(r_M,x)) < A(s+r_M+2,x)$$
il che dimostra che $f$ è maggiorata da $A$.

\paragraph{Ricorsione}
Supponiamo di avere $g^k,h^{k+2}$ maggiorate da $A$.
Questo significa che $g^k(\overline{x}) < A(r, x)$ e $h(\overline{y}) < A(s, y)$, per opportuni valori $r,s$.
Sia allora $f^{k+1} = R[g^k,h^{k+2}]$; iniziamo dimostrando che $f^{k+1}(\overline{x}, n) < A(q,n+x)$ per un valore $q$ non dipendente da $n$ né da $x$.
Questa è una facile dimostrazione per induzione su $n$, una volta detto $q=1+max(r,s)$.
$$f(\overline{x}, 0) = g(\overline{x}) < A(r,x) < A(q,x)$$
\begin{align*}
  f(\overline{x}, n+1) &= h(\overline{x}, n, f(\overline{x},n)) \\
                       &< A(s,max(\overline{x}, n, f(\overline{x},n))) \\
                       &< A(s, A(q,n+x)) \\
                       & \leq A(q-1, A(q,n+x)) \\
                       &= A(q,n+1+x)
\end{align*}

A questo punto, dati $x,y\in \mathbb{N}$, e detto $m$ il loro massimo, abbiamo che:
$$f(x,y) < A(q,x+y) \leq A(q,2m) < A(q,2m+3) = A(q,A(2,m)) = A(q+4,z)$$
il che dimostra che $f$ è maggiorata da $A$.

\paragraph{Conclusione}
Visto che sappiamo che l'insieme delle funzioni primitive ricorsive è il minimo insieme che contiene le funzioni iniziali, ed è chiuso per composizione e ricorsione primitiva, abbiamo che tutte le funzioni primitive ricorsive sono maggiorate da $A$. Per quanto visto all'inizio, $A$ non può allora essere una funzione primitiva ricorsiva.

\section*{Esercizio 5}
\paragraph{Testo}
Produttività per $n=5$.

\section*{Esercizio 6}
\paragraph{Testo}
Dimostrare che utilizzando la $\eta$-uguaglianza $\lambda x.M(x) = M$ , con $x \notin FV(M)$, si può dedurre $M(x) = L(x) \Rightarrow M = L$ se $x \notin F V (M ) \cup F V (L)$.

\paragraph{Soluzione}
Gli ingredienti necessari per questa soluzione sono gli assiomi del $\lambda$-calcolo detti $\xi$-equivalenza e $\eta$-equivalenza, riportati qui di seguito:
\begin{center}
  \AxiomC{$b=d:term$}
  \AxiomC{$x:var$}
  \BinaryInfC{$(\lambda x.b) = (\lambda x.d) : term$}
  \begin{equation}\tag{$\xi$-equivalenza} \DisplayProof \end{equation}
\end{center}
\begin{center}
  \AxiomC{$b:term$}
  \AxiomC{$x:var$}
  \RightLabel{se $x \notin \mathbb{FV}(b)$}
  \BinaryInfC{$\lambda x.b(x) = b : term$}
  \begin{equation}\tag{$\eta$-equivalenza}  \DisplayProof  \end{equation}
\end{center}

Una derivazione del risultato sotto le ipotesi date è quindi:

\AxiomC{$M(x)=L(x):term$}
\AxiomC{$x:var$}
\BinaryInfC{$\lambda x.M(x) = \lambda x.L(x) : term$}

\AxiomC{$M:term$}
\AxiomC{$x:var$}
\BinaryInfC{$\lambda x.M(x) = M : term$}

\AxiomC{$L:term$}
\AxiomC{$x:var$}
\BinaryInfC{$\lambda x.L(x) = L : term$}

\TrinaryInfC{$M=L : term$}
\begin{equation*} \DisplayProof \end{equation*}

\section*{Esercizio 7}
\paragraph{Testo}
Mostrare che i due modi per definire la chiusura riflessiva e transitiva sono equivalenti.

\paragraph{Soluzione}
Vogliamo dimostrare, in particolare, che è la stessa cosa definire $\twoheadrightarrow_{\beta}$ secondo gli schemi:
\begin{center}
  $M \twoheadrightarrow_{\beta} M$ \qquad
  \AxiomC{$M \rightarrow_{\beta} N$}
  \UnaryInfC{$M \twoheadrightarrow_{\beta} N$}
  \DisplayProof \qquad
  \AxiomC{$M \twoheadrightarrow_{\beta} N$}
  \AxiomC{$N \twoheadrightarrow_{\beta} L$}
  \BinaryInfC{$M \twoheadrightarrow_{\beta} L$}
  \DisplayProof
\end{center}
oppure dicendo che è l'intersezione di tutte le relazioni riflessive e transitive che contengono $\rightarrow_{\beta}$.

Chiamiamo $\mathcal{R}$ l'insieme delle relazioni riflessive e transitive che contengono $\rightarrow_{\beta}$.
Per definizione, abbiamo che $\twoheadrightarrow_{\beta} \in \mathcal{R}$, e allora $\twoheadrightarrow_{\beta} \subseteq\bigcap_{R\in \mathcal{R}} R$, che è una parte dell'equivalenza.

Per l'altra, ovvero che se $M \twoheadrightarrow_{\beta} N$ allora $\forall R \in \mathcal{R}, \quad M R N$,
iniziamo notando che, se due termini sono uguali (e quindi $M \twoheadrightarrow_{\beta} M$), $MRM$ è sempre vero, perché tutte le relazioni in $\mathcal{R}$ sono riflessive.

Per due termini distinti $M$ ed $N$, sappiamo che, se $M \twoheadrightarrow_{\beta} N$, allora esiste una catena di termini $L_0,\dots,L_n$ tale che $L_0=M$, $L_n=N$ e si ha $L_i \rightarrow_{\beta} L_{i+1}$ per $i=0,\dots, n-1$ (possiamo sempre dimostrarlo per induzione strutturale sulla lunghezza della prova).
Allora, data una relazione $R$, visto che $\rightarrow_{\beta} \subseteq R$, abbiamo che $L_i R L_{i+1}$. Dimostriamo allora che $L_0RL_j$, per induzione su $j$.
Il caso base è ovvio ($L_0RL_1$). Il caso induttivo si ottiene dall'uso dell'ipotesi induttiva e dalla transitività della relazione ($L_0RL_{j-1}$ e $L_{j-1}RL_j$ implicano $L_0RL_j$.
Ne deduciamo quindi che $L_0RL_n$, ovvero che $MRN$, e la nostra tesi è dimostrata per la generalità di $R$. 



\section*{Esercizio 8}
\paragraph{Testo}
Rivedere il Teorema di Tarski per trovare un massimo punto fisso.

\section*{Esercizio 9}
\paragraph{Testo}
Mostrare che se $x = y$ e $x \in FV(L)$ allora $M [x := N ][y := L] = M [y := L][x :=N [y := L]]$.

\section*{Esercizio 10}
\paragraph{Testo}
Dimostrare il seguente lemma per induzione sulla complessità di $M$: Se $M \rightarrow_1 \overline{M}$ e $N \rightarrow_1 \overline{N}$ allora $M [x := N] \rightarrow_1 \overline{M} [x := \overline{N}]$.

\section*{Esercizio 11}
\paragraph{Testo}
Dimostrare la proprietà del diamante per $\rightarrow_1$ per induzione sulla complessità del termine.

\section*{Esercizio 12}
\paragraph{Testo}
Cosa succede al teorema di rappresentabilità se passo da $\Lambda$ a $\Lambda^\rightarrow$? Quali funzioni calcolabili sono compilabili con la macchina $\Lambda^\rightarrow$?

\section*{Esercizio 13}
\paragraph{Testo}
Dimostrare che qualunque funzione totale da $Boole^n \rightarrow Boole$ può essere scritta usando solo IfThenElse.

\section*{Esercizio 14}
\paragraph{Testo}
Fornire l’algoritmo che, dato un $\lambda$-termine in $\Lambda^\rightarrow$, fornisca tipo e contesto validi e si blocchi quando ciò non è possibile. $? \vdash M :\: ?$.

\section*{Esercizio 15}
\paragraph{Testo}
Fornire l’algoritmo che, dati contesto e tipo, fornisca un $\lambda$-termine valido e si blocchi quando ciò non è possibile. $\Gamma \vdash \: ? : \alpha$.

\section*{Esercizio 16}
\paragraph{Testo}
Compila $(\lambda x.\lambda y. x+y)(1)(1)$ e verifica che è uguale a due.

\section*{Esercizio 17}
\paragraph{Testo}
Dimostrare che unione arbitraria di saturi è un saturo e intersezione arbitraria di saturi è un saturo.

\section*{Esercizio 18}
\paragraph{Testo}
Dimostrare tutto ciò che serve per il teorema di validità per il tipo delle coppie.

\section*{Esercizio 19}
\paragraph{Testo}
Dimostrare che se $X \subseteq F N$ allora esiste un minimo insieme saturo $S$ tale che $X \subseteq S$.

\section*{Esercizio 20}
\paragraph{Testo}
Mostrare che in $\llbracket \alpha \times \beta \rrbracket^{MIN} \subseteq \llbracket \alpha \times \beta \rrbracket^{MAX}$ l’inclusione è stretta, anzi che esiste un saturo $S$ tale che $\llbracket \alpha \times \beta \rrbracket^{MIN} \subset S \subset \llbracket \alpha \times \beta \rrbracket^{MAX}$.

\section*{Esercizio 21}
\paragraph{Testo}
Estendere il Teorema di Church-Rosser da $\Lambda^\rightarrow$ a $\Lambda^{\rightarrow, \times}$.

\section*{Esercizio 22}
\paragraph{Testo}
Dimostrare il teorema di validità per il tipo $\alpha \varoplus \beta$ disgiunzione.

\section*{Esercizio 23}
\paragraph{Testo}
Inventarsi un tipo nuovo (esempio: $\alpha \times X$, $X \times X$, $List(\alpha \times X)$, $\dots$) e far vedere che funziona.

\section*{Esercizio 24}
\paragraph{Testo}
Scrivere la funzione $eqnat : (\mathbb{N} \times \mathbb{N}) \rightarrow Boole$ che traduca la nozione di uguaglianza sui naturali.

\section*{Esercizio 25}
\paragraph{Testo}
Implementare le seguenti funzioni
$$\begin{cases} F(0,y) = K(y) \\ F(succ(x),y) = g(x,y,F(x,y)) \end{cases}$$
$$\begin{cases} F(0,y) = K(y) \\ F(succ(x),y) = g(x,y,F(x,d(y))) \end{cases}$$
e usarle per definire $Eqz(x,y)$.

\section*{Esercizio 26}
\paragraph{Testo}
Mostrare, nel caso delle liste, che le funzioni definite con la recursion si possono definire con la tail-recursion.

\section*{Esercizio 27}
\paragraph{Testo}
Definire le funzioni:
\begin{itemize}
\item $Append : List(A) \times List(A) \rightarrow List(A)$ che concatena due liste;
\item $Reverse : List(A) \rightarrow List(A)$ che ribalta una lista;
\item $Map : (\alpha \rightarrow \beta) \times List(A) \rightarrow List(\beta)$ che applichi una funzione $f$ ad ogni entrata della lista;
\item $Filter : (\alpha \rightarrow Boole) \times List(\alpha) \rightarrow List(\alpha)$ che toglie dalla lista gli elementi che vanno in falso;
\item $List(N) \rightarrow List(N)$ che prende liste di naturali e li ordina.
\end{itemize}

\section*{Esercizio 28}
\paragraph{Testo}
Usare un albero binario per ordinare una lista [$heapsort$]. 

\section*{Esercizio 29}
\paragraph{Testo}
Dare la regola di eliminazione per il tipo $B(\alpha)$ per alberi binari avente le seguenti regole di introduzione:
\begin{center}
\begin{tabular}{p{4cm} p{4cm}}
  \AxiomC{$\Gamma \vdash a : \alpha$}
  \UnaryInfC{$\Gamma \vdash leaf(a) : B(\alpha)$}
  \DisplayProof
  &
  \AxiomC{$\Gamma \vdash t_1 : B(\alpha)$}
  \AxiomC{$\Gamma \vdash t_2 : B(\alpha)$}
  \BinaryInfC{$\Gamma \vdash bin(t_1,t_2) : B(\alpha)$}
  \DisplayProof
\end{tabular}
\end{center}
Dare le interpretazioni per $leaf(a)$ e $bin(t_1, t_2)$, oltre all’interpretazione massima e minima di $B(a)$. Dedurne il teorema di validità.

\section*{Esercizio 30}
\paragraph{Testo}
Si considerino le seguenti regole di introduzione per un tipo $T(\alpha)$:
\begin{center}\begin{tabular}{p{3cm} p{3cm}}
  $\vdash_\triangle : T(\alpha)$
  &
  \AxiomC{$\Gamma \vdash g : \alpha \times T(\alpha)$}
  \UnaryInfC{$\Gamma \vdash \square(g) : T(\alpha)$}
  \DisplayProof
\end{tabular}
\end{center}

Si definisca un’opportuna nozione di isomorfismo di tipi, e rispetto a tale nozione si dimostri che $T(\alpha)$ è isomorfo a $List(\alpha)$.

\section*{Esercizio 31}
\paragraph{Testo}
Definire il nuovo tipo $Tree_n(\alpha)$ di alberi in cui ogni nodo ha esattamente $n$ figli.

\section*{Esercizio 32}
\paragraph{Testo}
Definire il nuovo tipo $BT_\ast (\alpha)$ di alberi in cui ogni nodo ha al più $2$ figli.

\section*{Esercizio 33}
\paragraph{Testo}
Dare un algoritmo che, dati due alberi binari ovvero di tipo $BT(\alpha)$ dica se sono equivalenti per vermi e che termini non appena le liste di elementi differiscano.

\section*{Esercizio 34}
\paragraph{Testo}
Definire la funzione $d(\theta, n) : Ord(\mathbb{N}) \times \mathbb{N} \rightarrow \mathbb{N}$ che associa all’ordinale $\theta$ la lunghezza del cammino che ad ogni ordinale limite sceglie l’ennesimo percorso.

\section*{Esercizio 35}
\paragraph{Testo}
Dimostrare che non esiste alcuna funzione suriettiva tipabile $h : \mathbb{N} \rightarrow Ord(\mathbb{N})$.

\section*{Esercizio 36}
\paragraph{Testo}
Definire i tipi $\alpha \times T(\alpha)$ e $\alpha \varoplus T(\alpha)$.

\section*{Esercizio 37}
\paragraph{Testo}
Mostrare in casi specifici, a scelta, che le regole di introduzione sono valide, (esempio fatto in classe: $n \in \mathbb{N} \rightarrow succ(n) \in \mathbb{N}$), e che lo sono anche le regole di eliminazione.

\section*{Esercizio 38}
\paragraph{Testo}
Dimostrare il lemma di sostituzione in $\Lambda^{\rightarrow, \Pi, \mu}$ per induzione sulla complessità del termine, ovvero che $\llbracket \tau [x := \alpha]\rrbracket_\sigma = \llbracket \tau \rrbracket_{\sigma(x|\llbracket \alpha \rrbracket_\sigma)}$.
\end{document}
